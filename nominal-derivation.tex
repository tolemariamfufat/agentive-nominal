\documentclass[11pt,a4paper]{article}
\usepackage{tipa}
\usepackage{apacite,pslatex}
\usepackage{titling}
\title {Nominal Derivaton in Afaan Oromoo}
\author {Tolemariam Fufa}
\date{September 2022}
\begin{document}
\newcommand\keywords[1]{%
	\begingroup
	\let\and\\
	\par
	\noindent\textbf{Keywords:}\\#1\par
	\endgroup
}
\maketitle
\begin{abstract}
	
	Afaan Oromoo Nominal derivation is not thoroughly treated in the literature. In fact, \cite{temesgen1985} mentioned some nominal derivations such as result nominals. Yet, he doesn’t determine range of variations of nominal derivations in terms of type, form and meaning. This paper is aimed to fill such research gap. The paper is divided into three parts. The first part is dedicated to identify describe agent nominals in terms of definition, type, form and meaning in Afaan Oromoo. This section discusses agent nominal in relation to semantic roles because semantic roles relate verbs with arguments; this is important as agent nominals are derived from dynamic verbs. It also describes the three types of agent nominals: The doer of an act, agents with a functional status and behavioral agent by citing Afaan Oromoo examples.The second section is devoted to event nominals. This section defines event nominals interms of argument structure and derivatioal patterns. It also considers form and meaning of event nominals of Afaan Oromoo. The third section discusses result nominals. The section defines result nominals. The section discusses whether or not result nominals differ from event nominals. The section also describes that result nominals are referential in the sense that they do not relevant to semantic roles \cite{melloni2012event}. Moreover, the section discusses forms of result nominals, specifically from which root or base result nominals derived. The section also discusses meaning ranges of result nominals.\\
		
	
\end{abstract}
\keywords{nominal, derivation, agent, event, result}
\newpage
\section{Introduction}
In Afaan Oromoo nominals can be derived from different word classes. Many abstract nouns are derived from nouns by affixing morphemes such as -ummaa \cite{gaddisa2001,temesgen1985,tolemariam2011}. For instance the following  abstract nominals are derived in such manner. \\

(1)\\
\indent nam-ummaa \\
\indent man-N\\
\indent 'humanity'\\

(2)\\
\indent garb-ummaa\\
\indent slave-N\\
\indent 'slavery'\\

(3)\\
\indent dubar-ummaa\\
\indent female-N\\
\indent 'feminity'\\

As shown in these examples abstract nominals namummaa 'humanity' garbummaa 'slavery' and dubartummaa 'feminity' are derived from nouns nama 'man', garba 'slave' and dubartii 'female' respectively. 

In a similar way, nouns can be derived from adjectival bases in Afaan Oromoo \cite{gaddisa2001,temesgen1985,tolemariam2011,tolemariam2009}. \\

(4) \\
\indent gaar-ummaa\\
\indent good-N\\
\indent 'goodness'\\

(5)\\
\indent diim-ina\\
\indent red-N\\
\indent 'redness'\\

(6)\\
\indent hamm-ina\\
\indent cruel-N\\
\indent 'cruelty'\\

As shown above abstract nominals gaarummaa 'goodness', diimina 'redness' and hammina 'cruelty' are all derived from adjectival bases. 

The main focus of this paper is deverbial nominals. In Afaan Orommoo deverbial nominals are also abundant. Nominals can be derived from intransitive roots. Nominals can be derived from transitive bases. In this language, different nominal categories can be derived from derived verbs such as causative, middle and passive. In particular, the paper focuses on four deverbial nominal derivations. These deverbial nominal derivations are agent, event, result and patient nominals. 

The paper is amied to answer the following research questions:\\
\begin{enumerate}
	\item What criterion do we use to identify agent, event, result and patient nominals? 
	\item From which verb stems are agent, event, result and patient nominals derived?
	\item Which nominal affixes are employed to derive agent, event, result and patient nominals?
\end{enumerate}

The nature of this research is qualitative. Data is gathered from different Afaan Oromoo books and MA theses. I used Linguistic Latex \footnote{LaTeX creates papers, presentations and artiles widely used in linguistics and other scientific researches.} free software tool for write up. 

\section{Agent Nominal}

In linguistic literature agent nominals are nominals derived from agenive vebrbs; they describe performers of an action \cite{huyghe2020s}. In fact, the concept agent is difficult to define. Major properties  such as animacy, control, volition, instigation, accountability, motion, etc have been long outlined and debated in the literature. Animacy, volition, and instigation has never been a global criterion for agent. Some linguists claim that agentive refers to an animate entity which happen to be the willful source or agent of the activity described by the verb \cite{cruse1973some} while others argue against this concept and on the contrary they claim that intention is not a necessary criterion of agentivity for we often do things accidentally \cite{schlesinger1989instruments,caspar2015relationship}. For my discussion I follow \cite[p-188]{huyghe2020s} general definition of agents as effectors that are prototypically, but not necessarily, animate and intentional in the sense that the participant can be volitional or original instigator that brings about something about; it refers to an action performed by an entity which is considered as employing its own energy in carrying out the action. Hence, expriencers are not considered  agents since agents are assumed to be involved in dynamic situations. 


Agent nominals are described in terms of  semantic roles. Semantic roles explain the relationship between verbs and arguments. In this sense nouns can be categorized as agent nominals if they link their referent as the agent of intrinsically specified action \cite[p-189]{huyghe2020s}. 

Two conditions must be met for typology of agent nominal. One, the presence of dynamic action in the semantic structure; and two, the description of the the agent. In this case agent nominals are assumed to be deverbal nouns that denote the agentive argument of the base verb \cite[p-189-190]{huyghe2020s}.

The range of variation of prototypical agent nominals can be classified into 3 types. The doer of an act (agent nominals that denote agents involved in a specific event), the agent of an action ( ANs that denote agents with a functional status, can be used as bare predicates) and behavioral agent (that denote agents with a propensity to do certain things or to act in a certain way, repetition of actions) \cite[190-192]{huyghe2020s}. For example, ‘savior’  is doer of an act; ‘rescue worker’ is an agent of a function; while ‘fighter’, ‘hard worker’ , etc are considered as behavioral agents.

In many languages agent nominals are identified by agent morpheme that is attached to the base fo the nominal under consideration. For example in English morphemes sucha as -er (teacher), -ist (populist), an (veteran), eer (engineer) indicate agent nominals \cite[p-186]{huyghe2020s}.

In Afaan Oromoo there is no linguistic research dedicated to the description of agent nominals. Agent nominals in this language are identified by the moropheme -aa and -tuu. The morpheme -aa marks masculine whereas the morpheme -tuu marks feminine or dimunitive. In fact in Afaan Oromoo agent nominal derivation involves complex morphological processes. They are morophologically complex in the sense that they require causative or middle bases. \\

Doer of an act (derived from transitive roots/bases)\\

Agent nominals are underlyingly causers. Causer nominals are agent nominals. In Afaan Oromoo, agent nominals are created from corresponding verbs in a complex way in the sense that they involve morphological processes. Specifically, they require the causative affix to be attached (not in all cases) to which the agent nominal affix is get attached to. 

(1a)\\
\indent	 bar-e\\
\indent know-3MSS\\
\indent	‘He knew ...'\\

(1b) \\
\indent	bar-siis-aa/tuu\\
\indent know-CAUS-NM/F\\
\indent ‘teacher’\\

(2a) 	\\
\indent konkol:aat-e\\
\indent roll:MID-3MSS\\
\indent ‘rolled’\\

(2b) \\	
\indent konkol:aat-is-aa/tuu\\
\indent roll:MID-CAUS-NM/F\\
\indent ‘driver’\\

(2c) \\
\indent gudd-aa\\
\indent big-ADJ\\
\indent 'big'\\

(2d)\\
\indent gudd-is-aa/tuu\\
\indent big-CAUS-NM/F\\
\indent 'babysitter'\\

As shown in (1a) bar- ‘to know’ is the root word to which the causative morpheme – siis – is attached in (1b). agent nominal morpheme – aa – or – tuu – is suffixed to the causative base. 

There are also cases where more complex morphological processes involved. For example in the following instance three derivational suffixes involved in the creation of agent nominal.\\

(4a) 	\\
\indent fira ‘relative’\\

(4b) 	\\
\indent fir-oom-e\\
\indent relative-MID-3MSS\\
\indent ‘...became relative’\\

(4c)	\\
\indent fir-oom-s-e\\
\indent relative-MID-CAU-3MSS\\
\indent ‘...made sb become relative’\\

(4d)	\\
\indent fir-oom-s-aa\\
\indent relative-MID-CAUS-NM\\
\indent ‘Social bond creator (relativizer?)’\\

(4e)\\
\indent	fir-oom-s-ituu\\
\indent relative-MID-CAUS-NF\\
\indent ‘Social bond creator(relativizer?)’\\

Behavioral agent (derived from intransitive roots)\\

5) 	\\
\indent hat-e\\
\indent steal-3MSS\\
\indent ‘He stole’\\
6) 	\\
\indent hat-tuu\\
\indent steal-NF\\
\indent ‘Thief’\\

7a) \\
\indent	sob-e\\
\indent lie-3MSS\\
\indent ‘He lied’\\

7b)	\\
\indent sob-tuu/duu \\
\indent lie-NF\\
\indent ‘Liar’\\

\section{Event Nominal}
Event nominals are also known as action or process nominal in the literature \cite{siloni1997event}. Syntactically, Event nominals share thematic relations of the base verb; they inherit the argument structure of the base verb \cite{uth2015event,siloni1997event}. Semantically, they transpose the meaning of base verbs. Event nominals are “abstract nouns that ‘give a name’ to the situations (i.e.event or states) expressed by their corresponding predicates.”\cite{uth2015event}.

In Afaan Oromoo, event nominals are derived by mophemes  -aatii, -itii, \\

(1a) 	\\
\indent isaan 		karaa 		cuf-an.\\
\indent they.NOM	road.ABS	block-3PS.PER\\
\indent ‘They blocked the road.’\\

(1b) 	\\
\indent karaa		cuf-aatii	isaan-ii\\
\indent road.ABA	block-NOM	they-POSS\\
\indent “Their blockage of a road”\\

As shown in (1a) cuf-an ‘blocked’ is a transitive verb, karaa ‘road’ is object and isaan ‘they’ is subject. (1b) is a nominal phrase derived from (1a). In (1b), the head noun is cuf-aatii ‘blockage’. This nominal is an event nominal that is derived from base verb cuf- ‘block’. The nominalizing affix -aatii is attached to the base verb cuf- ‘block’ to derive the event nominal cuf-aatii. The phrase ‘karaa cufaatii isaanii’ gives name to the event described by verb cuf- ‘block’ in (1a). \\

(2a) 	\\
\indent kaleessa 		muka 		muran-an\\
\indent yesterday		barley		cut-3PS.PERF\\
\indent ‘They cut a tree yesterday.’\\


(2b)	\\
\indent muka		mur-aatii	kaleessa-a\\
\indent muka		mur-NOM	yesterday-POSS\\
\indent ‘Yesterday’s tree cutting ...’\\


(3a) 	\\
\indent nuti 		tulluu 		yaab-ne 	\\	
\indent we		mountain	climb-1PS\\
\indent ‘We climbed a mountain.’\\


(3b) \\
\indent Tulluu yaabb-ittii 	\\	
\indent Mountain climb-N	\\
\indent ‘Mountain climbing’\\

Event nominal derivation is not straight forward always. In some cases event nominal overlaps with result nominal as shown in the following example:\\

(5a)  \\
\indent Inni 	 biiraa 	\textipa{\!d}ug-e\\
\indent he 	 beer 	drink-3MSS\\
\indent ‘he drank beer’\indent

(5b) 	\\
\indent Biiraa 	\textipa{\!d}ugaa-tii\\
\indent Beer  	Drink-N\\
\indent‘Drinking beer’\\

(5c)\\
\indent	\textipa{\!d}ug-aatii\\
\indent (a) event nominal \\
\indent (b) referential (it means any kind or drink)\\

\section{Result Nominal}
As compared to agent and event nominals, result nominals are said to be non-argumental nominals. Result nominals are not explained in terms of semantic roles and verb-argument relationships. They are not formulated in the spirit of GB theory (Uth, 2015:4). Therefore, result nominals are treated from semantic point of view. That is to say the meanings of result nominals are not transposed to or from argument structure of the verb from which they have been derived. Uth claims that result nominals are purely  referential \cite[p-4]{uth2015event}. But my data shows that this is not the case. I came across result nominals which are referential as well as result nominas that are explined in terms of semantic roles. 

In Afaan Oromoo, many result nominals employ unique affixes such as -umsa and -ina \cite{gaddisa2001,temesgen1985,tolemariam2011}. The suffix -umsa is attached to root verb to derive result nominal as shown in (6c). This type of result nominals are purely referential. \\

(6) \\
\indent k'or-umsa\\
\indent exmine-N\\
\indent ‘examination’\\
\indent *k'or-at-umsa\\
\indent *k'or-siis-umsa\\
(7) \\
\indent jeek'-umsa\\
\indent distrub-N\\
\indent ‘distrubance’\\
\indent *jeek'-sis-umsa\\

(8)\\
\indent bar-umsa RESULT\\
\indent Know-N\\
\indent ‘education’\\
\indent *bar-siis-umsa\\
\indent *bar-at-umsa\\

(9)\\
\indent Beek-umsa\\
\indent Know-N\\
\indent ‘Knowledge’ Result\\
\indent *beek-sis-umsa\\ 


The morpheme -ina is attached to verbs orginated from adjectival roots. Such result nominals are said to be measure result nouns \cite{uth2015event}. The morpheme -ina can also be attached to intransitive and transitive roots to derive rusult nominals \cite{gaddisa2001,temesgen1985,tolemariam2011}. \\

(10)\\
\indent gudd-at-e\\
\indent big-MID-3MSS\\
\indent 'He became big'\\

(11)\\
\indent gudd-ina\\
\indent big-N\\
\indent 'growth'\\

(12)\\
\indent \textipa{\!d}eer-ina\\
\indent long-N\\
\indent 'length'\\

(13)\\
\indent bal\textipa\textbarglotstop-at-e\\
\indent wide-MID-3MSS\\
\indent 'width'\\

(14)\\
\indent bal\textipa\textbarglotstop-ina\\
\indent 'wide-N\\
\indent 'width'\\

The morpheme can be attached intransitive and transitive verb roots to derive result nominals \cite{gaddisa2001,temesgen1985,tolemariam2011}.\\

(15)\\
\indent fiig-e \\
\indent run - 3MSs\\
\indent 'He run'\\

(16)\\
\indent fiig-insa (fiig-ica)\\
\indent run -N\\
\indent 'run'\\

(17)\\
\indent diig-insa\\
\indent destroy-N\\
\indent 'destruction'\\

(18)\\
\indent k'ot-insa (k'otisa)\\
\indent farm-N\\
\indent 'farm'\\

\section{Patient Nominal}
Patient is the person or entity that is affected by the action described by a verb. The causer and the controller of the action is known as agent as we have discussed in section one. The following criterion can be considered for patient: undergoes change of state, incremental theme, causally affected by another participant and stationary relative to movement of another participant \cite{barker1993nominal}. 

In Afaan Oromoo patient nominals are often derived from middle or passive bases as opposed to agent nominals which are often drived from causative stems \cite{tolemariam2009}. In this language person names clearly show the comparison of agent and patient nominal derivations. We can dare to say for every agent nominal there is one patient nominal as far as proper nouns or person names concerned. \\


(19) AGENT \\
\indent Gudd-is-aa/tuu  AGENT\\
\indent big-CAUS-NM/F\\
\indent 'Guddisaa/One who makes it grow'\\

(20) PATIENT\\
\indent Gudd-at-aa/tuu \\
\indent big-MID-NM/F\\
\indent 'Guddataa/One who become big'\\

(21) AGENT\\
\indent Gabb-is-aa/ee \\
\indent fat-CAUS-NM/F\\
\indent 'Gabbisa/One who makes it fat'\\

(22) PATIENT\\
\indent Gabb-at-aa/tuu\\
\indent fat-MID-NM/F\\
\indent 'Gabbisaa/One who becomes fat'\\


(23)AGENT\\
\indent \textipa{\!d}eer-eess-aa \\
\indent long/tall-CAUS-N\\
\indent '\textipa{\!d}eeressaa'/One that makes it long/tall'\\

(24) PATIENT\\
\indent \textipa{\!d}eer-at-aa AGENT\\
\indent long/tall-CAUS-N\\
\indent '\textipa{\!d}eerataa'/One who becomes long/tall'

Professoion wise we can observe agent-patient contrast nominal derivations.


(25) AGENT \\
\indent bar-siis-aa/tuu \\
\indent know-CAUS-NM/F\\
\indent ‘Teacher’\\

(26) PATIENT \\
\indent bar-at-aa/tuu \\
\indent know-MID-NM/F\\
\indent ‘Student’\\

(27) AGENT \\
\indent hojj-at-iis-aa/hojj-aciis-aa \\
\indent work-MID-CAUS-N\\
\indent ‘Employer/Supervisor’\\

(28) PATIENT \\
\indent hojj-at-aa \\
\indent work-MID-N\\
\indent ‘worker’\\

Patient can also be derived from a passive stem.\\

(29) AGENT \\
\indent gor-s-aa/tuu \\
\indent advise-CAUS-NM/F\\
\indent ‘Advisor’\\

(30) PATIENT \\
\indent gor-s-am-aa/tuu\\
\indent advise-CAUS-PASS-NM/F\\
\indent ‘Advisee’\\ 



\section{Conclusion}

	Assumption 1: Agent, Event, Result and Patient nominals are deverbials.\\
Assumptioin 2: Derivational affixes such as passive, causative, transitivizer, middle create stems to which nominal affixes are attached\\
Assumption 3: Complex derivational suffixes sucha as CAUS + MID, MID + CAUS, MID + PASS, etc play roles in deriving agent , event adn patient nominals in Afaan Oromoo.
Assumption 4: Result nominals by virtue of their inherent referential character avoid verb derivation affixes such as CAUS, MID and PASS.
Assumption 5: Result nominal derivation require non-argument deverbial affixes. 
Assumption 6: I believe that analysis of derived nominal in terms of these affixes explains patterns of deverbal nominals. \\

\newpage
\bibliographystyle{apacite}
\bibliography{agent}

\newpage
\section*{Acronyms and symbols}
c=palatal affricate\\
CAUS=Causative\\
F=Feminine\\
M=Masculine\\
MID=Middle\\
N=Nominal\\
NF=Nominal Feminine\\
NM=Nominal Masculine\\
1PS=First Plural Subject\\
3MSS=Third Masculine Singular Subject\\
3PS.PERF=Third Plural Subject Perfect\\
POSS=Possessive\\
PASS=Passive\\
\end{document}