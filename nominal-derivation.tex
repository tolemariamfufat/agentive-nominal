\documentclass[11pt,a4paper]{article}
\usepackage{titling}
\title {Nominal Derivaton in Afaan Oromoo}
\author {Tolemariam Fufa}
\date{September 2022}
\begin{document}
\newcommand\keywords[1]{%
	\begingroup
	\let\and\\
	\par
	\noindent\textbf{Keywords:}\\#1\par
	\endgroup
}
\maketitle
\begin{abstract}
	Afaan Oromoo Nominal derivation is not thoroughly treated in the literature. In fact, Negasa (1995) mentioned some nominal derivations such as result nominals. Yet, he doesn’t determine range of variations of nominal derivations in terms of type, form and meaning. This paper is aimed to fill such research gap. The paper is divided into three parts. The first part is dedicated to identify describe agent nominals in terms of definition, type, form and meaning in Afaan Oromoo. This section discusses agent nominal in relation to semantic roles because semantic roles relate verbs with arguments; this is important as agent nominals are derived from dynamic verbs. It also describes the three types of agent nominals: The doer of an act, agents with a functional status and behavioral agent by citing Afaan Oromoo examples.The second section is devoted to event nominals. This section defines event nominals interms of argument structure and derivatioal patterns. It also considers form and meaning of event nominals of Afaan Oromoo. The third section discusses result nominals. The section defines result nominals. The section discusses whether or not result nominals differ from event nominals. The section also describes that result nominals are referential in the sense that they do not relevant to semantic roles(Melloni, 2012). Moreover, the section discusses forms of result nominals, specifically from which root or base result nominals derived. The section also discusses meaning ranges of result nominals.
		
\end{abstract}
\keywords{nominal \and derivation \and agent \and event \and result}
\newpage
\section{Agent Nominal}
In linguistic literature agent nominals are nominals derived from agenive vebrbs; they describe performers of an action (Huyghe,et.al 2020). In fact, the concept agent is difficult to define. Major properties  such as animacy, control, volition, instigation, accountability, motion, etc have been long outlined and debated in the literature. For my discussion I follow Huyghe, et,al (2020: 188) general definition of agents as effectors that are prototypically, but not necessarily, animate and intentional in the sense that the participant can be volitional or original instigator that brings about something about; it refers to an action performed by an entity which is considered as employing its own energy in carrying out the action. Hence, expriencers are not considered  agents since agents are assumed to be involved in dynamic situations. 

Agent nominals are described in terms of  semantic roles. Semantic roles explain the relationship between verbs and arguments. In this sense nouns can be categorized as agent nominals if they link their referent as the agent of intrinsically specified action (Huyghe, et,al 2020: 189) . 

Two conditions must be met for typology of agent nominal. One, the presence of dynamic action in the semantic structure; and two, the description of the the agent. In this case agent nominals are assumed to be deverbal nouns that denote the agentive argument of the base verb (Huyghe, et,al 2020: 189-190).

The range of variation of prototypical agent nominals can be classified into 3 types. The doer of an act (agent nominals that denote agents involved in a specific event), the agent of an action ( ANs that denote agents with a functional status, can be used as bare predicates) and behavioral agent (that denote agents with a propensity to do certain things or to act in a certain way, repetition of actions) Huyghe, et,al 2020: 190-192). For example, ‘savior’  is doer of an act; ‘rescue worker’ is an agent of a function; while ‘fighter’, ‘hard worker’ , etc are considered as behavioral agents.

In many languages agent nominals are identified by agent morpheme that is attached to the base fo the nominal under consideration. For example in English morphemes sucha as -er (teacher), -ist (populist), an (veteran), eer (engineer) indicate agent nominals (Huyghe,et.al 2020: 186).

In Afaan Oromoo there is no linguistic research dedicated to the description of agent nominals. Agent nominals in this language are identified by the moropheme -aa and -tuu. The morpheme -aa marks masculine whereas the morpheme -tuu marks feminine or dimunitive. In fact in Afaan Oromoo agent nominal derivation involves complex morphological processes. They are morophologically complex in the sense that they require causative or middle bases. \\

Doer of an act (derived from transitive roots/bases)\\

(1a)\\
\indent	 ajjees-e\\
\indent kill-3MSS\\
\indent	‘He killed ...'\\

(1b) \\
\indent	ajjees-aa\\
\indent kill-M\\
\indent ‘killer’\\

(2a) 	\\
\indent ijaar-e\\
\indent build-3MSS\\
\indent ‘built’\\

(2b) \\	
\indent ijaar-aa/tuu\\
\indent build-M/F\\
\indent ‘builder’\\

(2c) \\
\indent *Inni ijaaraadha.\\


Agent of a function (facilitates the condition), derived from double causatives\\

(3a)\\
\indent	bare-e\\
\indent know-3MSS\\
\indent ‘He knew …’\\

(3b)	\\
\indent bar-siis-aa/tuu\\
\indent know-CAUS-M/F\\
\indent ‘Teacher’\\

(3d) \\
\indent	Inni barsiisaadha. \\

(3e)	\\
\indent	Isheen barsiistuudha.\\

As shown in (1a) bar- ‘to know’ is the root word to which the causative morpheme – siis – is attached in (1b). agent nominal morpheme – aa – or – tuu – is suffixed to the causative base as shown in (1c-d). 

There are also cases where more complex morphological processes involved. For example in the following instance three derivational suffixes involved in the creation of agent nominal.\\

(4a) 	\\
\indent fira ‘relative’\\

(4b) 	\\
\indent fir-oom-e\\
\indent relative-MID-3MSS\\
\indent ‘...became relative’\\

(4c)	\\
\indent fir-oom-s-e\\
\indent relative-MID-CAU-3MSS\\
\indent ‘...made sb become relative’\\

(4d)	\\
\indent fir-oom-s-aa\\
\indent relative-MID-CAUS-M\\
\indent ‘Social bond creator (relativizer?)’\\

(4e)\\
\indent	fir-oom-s-ituu\\
\indent relative-MID-CAUS-F\\
\indent ‘Social bond creator(relativizer?)’\\

Behavioral agent (derived from intransitive roots)\\

5) 	\\
\indent hat-e
\indent steal-3MSS
\indent ‘He stole’
6) 	\\
\indent hat-tuu
\indent steal-F
\indent ‘Thief’

7a) \\
\indent	sob-e\\
\indent lie-3MSS\\
\indent ‘He lied’\\

7b)	\\
\indent sob-tuu/duu \\
\indent lie-F\\
\indent ‘Liar’\\


\end{document}