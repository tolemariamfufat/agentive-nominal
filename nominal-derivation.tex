\documentclass[11pt,a4paper]{article}
\usepackage{titling}
\title {Nominal Derivaton in Afaan Oromoo}
\author {Tolemariam Fufa}
\date{September 2022}
\begin{document}
\newcommand\keywords[1]{%
	\begingroup
	\let\and\\
	\par
	\noindent\textbf{Keywords:}\\#1\par
	\endgroup
}
\maketitle
\begin{abstract}
	Afaan Oromoo Nominal derivation is not thoroughly treated in the literature. In fact, Negasa (1995) mentioned some nominal derivations such as result nominals. Yet, he doesn’t determine range of variations of nominal derivations in terms of type, form and meaning. This paper is aimed to fill such research gap. The paper is divided into three parts. The first part is dedicated to identify describe agent nominals in terms of definition, type, form and meaning in Afaan Oromoo. This section discusses agent nominal in relation to semantic roles because semantic roles relate verbs with arguments; this is important as agent nominals are derived from dynamic verbs. It also describes the three types of agent nominals: The doer of an act, agents with a functional status and behavioral agent by citing Afaan Oromoo examples.The second section is devoted to event nominals. This section defines event nominals interms of argument structure and derivatioal patterns. It also considers form and meaning of event nominals of Afaan Oromoo. The third section discusses result nominals. The section defines result nominals. The section discusses whether or not result nominals differ from event nominals. The section also describes that result nominals are referential in the sense that they do not relevant to semantic roles(Melloni, 2012). Moreover, the section discusses forms of result nominals, specifically from which root or base result nominals derived. The section also discusses meaning ranges of result nominals.\\
		
	Assumption 1: Agent, Event and Result nominals are deverbials.\\
	Assumptioin 2: Derivational affixes such as passive, causative, transitivizer, middle create stems to which nominal affixes are attached\\
	Assumption 3: I believe that analysis of derived nominal in terms of these affixes explains patterns of deverbal nominals. \\
	
\end{abstract}
\keywords{nominal, derivation, agent, event, result}
\newpage
\section{Agent Nominal}

In linguistic literature agent nominals are nominals derived from agenive vebrbs; they describe performers of an action (Huyghe,et.al 2020). In fact, the concept agent is difficult to define. Major properties  such as animacy, control, volition, instigation, accountability, motion, etc have been long outlined and debated in the literature. Animacy, volition, and instigation has never been a global criterion for agent. Some linguists claim that agentive refers to an animate entity which happen to be the willful source or agent of the activity described by the verb (Gruber, 1967:943) while others argue against this concept and on the contrary they claim that intention is not a necessary criterion of agentivity for we often do things accidentallySchlesinger (1989:194; Kasper,et.al,2010). For my discussion I follow Huyghe, et,al (2020: 188) general definition of agents as effectors that are prototypically, but not necessarily, animate and intentional in the sense that the participant can be volitional or original instigator that brings about something about; it refers to an action performed by an entity which is considered as employing its own energy in carrying out the action. Hence, expriencers are not considered  agents since agents are assumed to be involved in dynamic situations. 


Agent nominals are described in terms of  semantic roles. Semantic roles explain the relationship between verbs and arguments. In this sense nouns can be categorized as agent nominals if they link their referent as the agent of intrinsically specified action (Huyghe, et,al 2020: 189) . 

Two conditions must be met for typology of agent nominal. One, the presence of dynamic action in the semantic structure; and two, the description of the the agent. In this case agent nominals are assumed to be deverbal nouns that denote the agentive argument of the base verb (Huyghe, et,al 2020: 189-190).

The range of variation of prototypical agent nominals can be classified into 3 types. The doer of an act (agent nominals that denote agents involved in a specific event), the agent of an action ( ANs that denote agents with a functional status, can be used as bare predicates) and behavioral agent (that denote agents with a propensity to do certain things or to act in a certain way, repetition of actions) Huyghe, et,al 2020: 190-192). For example, ‘savior’  is doer of an act; ‘rescue worker’ is an agent of a function; while ‘fighter’, ‘hard worker’ , etc are considered as behavioral agents.

In many languages agent nominals are identified by agent morpheme that is attached to the base fo the nominal under consideration. For example in English morphemes sucha as -er (teacher), -ist (populist), an (veteran), eer (engineer) indicate agent nominals (Huyghe,et.al 2020: 186).

In Afaan Oromoo there is no linguistic research dedicated to the description of agent nominals. Agent nominals in this language are identified by the moropheme -aa and -tuu. The morpheme -aa marks masculine whereas the morpheme -tuu marks feminine or dimunitive. In fact in Afaan Oromoo agent nominal derivation involves complex morphological processes. They are morophologically complex in the sense that they require causative or middle bases. \\

Doer of an act (derived from transitive roots/bases)\\

Agent nominals are underlyingly causers. Causer nominals are agent nominals. In Afaan Oromoo, agent nominals are created from corresponding verbs in a complex way in the sense that they involve morphological processes. Specifically, they require the causative affix to be attached (not in all cases) to which the agent nominal affix is get attached to. 

(1a)\\
\indent	 bar-e\\
\indent know-3MSS\\
\indent	‘He knew ...'\\

(1b) \\
\indent	bar-siis-aa/tuu\\
\indent know-CAUS-M/F\\
\indent ‘teacher’\\

(2a) 	\\
\indent konkol:aat-e\\
\indent roll:MID-3MSS\\
\indent ‘rolled’\\

(2b) \\	
\indent konkol:aat-is-aa/tuu\\
\indent roll:MID-CAUS-M/F\\
\indent ‘driver’\\

(2c) \\
\indent gudd-aa\\
\indent big-ADJ\\
\indent 'big'\\

(2d)\\
\indent gudd-is-aa/tuu\\
\indent big-CAUS-M/F\\
\indent 'babysitter'\\

As shown in (1a) bar- ‘to know’ is the root word to which the causative morpheme – siis – is attached in (1b). agent nominal morpheme – aa – or – tuu – is suffixed to the causative base. 

There are also cases where more complex morphological processes involved. For example in the following instance three derivational suffixes involved in the creation of agent nominal.\\

(4a) 	\\
\indent fira ‘relative’\\

(4b) 	\\
\indent fir-oom-e\\
\indent relative-MID-3MSS\\
\indent ‘...became relative’\\

(4c)	\\
\indent fir-oom-s-e\\
\indent relative-MID-CAU-3MSS\\
\indent ‘...made sb become relative’\\

(4d)	\\
\indent fir-oom-s-aa\\
\indent relative-MID-CAUS-M\\
\indent ‘Social bond creator (relativizer?)’\\

(4e)\\
\indent	fir-oom-s-ituu\\
\indent relative-MID-CAUS-F\\
\indent ‘Social bond creator(relativizer?)’\\

Behavioral agent (derived from intransitive roots)\\

5) 	\\
\indent hat-e\\
\indent steal-3MSS\\
\indent ‘He stole’\\
6) 	\\
\indent hat-tuu\\
\indent steal-F\\
\indent ‘Thief’\\

7a) \\
\indent	sob-e\\
\indent lie-3MSS\\
\indent ‘He lied’\\

7b)	\\
\indent sob-tuu/duu \\
\indent lie-F\\
\indent ‘Liar’\\

\section{Event Nominal}
Event nominals are also known as action or process nominal in the literature (Siloni 1997: 65). Syntactically, Event nominals share thematic relations of the base verb; they inherit the argument structure of the base verb (Uth, 2015:4; Siloni 1997:65). Semantically, they transpose the meaning of base verbs. As Uth (2015:4) puts Event are “abstract nouns that ‘give a name’ to the situations (i.e.event or states) expressed by their corresponding predicates.”

In Afaan Oromoo, event nominals are derived by mophemes  -aatii, -itii, \\

(1a) 	\\
\indent isaan 		karaa 		cuf-an.\\
\indent they.NOM	road.ABS	block-3PS.PER\\
\indent ‘They blocked the road.’\\

(1b) 	\\
\indent karaa		cuf-aatii	isaan-ii\\
\indent road.ABA	block-NOM	they-POSS\\
\indent “Their blockage of a road”\\

As shown in (1a) cuf-an ‘blocked’ is a transitive verb, karaa ‘road’ is object and isaan ‘they’ is subject. (1b) is a nominal phrase derived from (1a). In (1b), the head noun is cuf-aatii ‘blockage’. This nominal is an event nominal that is derived from base verb cuf- ‘block’. The nominalizing affix -aatii is attached to the base verb cuf- ‘block’ to derive the event nominal cuf-aatii. The phrase ‘karaa cufaatii isaanii’ gives name to the event described by verb cuf- ‘block’ in (1a). \\

(2a) 	\\
\indent kaleessa 		muka 		muran-an\\
\indent yesterday		barley		cut-3PS.PERF\\
\indent ‘They cut a tree yesterday.’\\


(2b)	\\
\indent muka		mur-aatii	kaleessa-a\\
\indent muka		mur-NOM	yesterday-POSS\\
\indent ‘Yesterday’s tree cutting ...’\\


(3a) 	\\
\indent nuti 		tulluu 		yaab-ne 	\\	
\indent we		mountain	climb-1PS\\
\indent ‘We climbed a mountain.’\\


(3b) \\
\indent Tulluu yaabb-ittii 	\\	
\indent Mountain climb-NOM	\\
\indent ‘Mountain climbing’\\

Event nominal derivation is not straight forward always. In some cases event nominal overlaps with result nominal as shown in the following example:\\

(5a)  \\
\indent Inni 	 biiraa 	dhug-e\\
\indent he 	 beer 	drink-3MSS\\
\indent ‘he drank beer’\indent

(5b) 	\\
\indent Biiraa 	Dhugaa-tii\\
\indent Beer  	Drink-NOM\\
\indent‘Drinking beer’\\

(5c)\\
\indent	dhug-aatii\\
\indent (a) event nominal \\
\indent (b) referential (it means any kind or drink)\\

\section{Result Nominal}
As compared to agent and event nominals, result nominals are said to be non-argumental nominals. Result nominals are not explained in terms of semantic roles and verb-argument relationships. They are not formulated in the spirit of GB theory (Uth, 2015:4). Therefore, result nominals are treated from semantic point of view. That is to say the meanings of result nominals are not transposed to or from argument structure of the verb from which they have been derived. Uth (2015:4) claims that result nominals are purely  referential. But my data shows that this is not the case. I came across result nominals which are referential as well as result nominas that are explined in terms of semantic roles. 

In Afaan Oromoo, many result nominals employ unique affixes. One such affix is -umsa. Another affix used in the derivation of result nominals is -at.  

The suffix -umsa is attached to root verb to derive result nominal as shown in (6c). This type of result nominals are purely referential. \\

6a) \\
\indent bar-siis-aa AGENT\\
\indent know-CAUS-N\\
\indent ‘Teacher’\\

(6b) \\
\indent bar-siis-a EVENT/Process\\
\indent know-CAUS-N\\
\indent ‘Behaviour’\\

(6c)\\
\indent bar-umsa RESULT\\
\indent Know-N\\
\indent ‘Knowledge’\\


(7a) \\
\indent beek-sis-aa AGENT\\
\indent know-CAUS-N\\
\indent ‘Notifier’\\

(7b)\\
\indent beek-sis-a EVENT\\
\indent know-CAUS-N\\
\indent ‘Notfication’\\

(7c)\\
\indent Beek-umsa\\
\indent Know-N\\
\indent ‘Knowledge’ Result\\

The morpheme – at – is employed in the nominal derivations used as proper nouns.  This type of result nominals are explained interms or thematic relationships because  of the presence of the middle – at -. This pattern is common in traditional personal names as shown below:

(8)\\
\indent Bal’-at-aa\\
\indent wide-MID-N\\
\indent ‘Bal’ataa/’extended’\\

(9)\\
\indent Gudd-at-aa\\
\indent big-MID-N\\
\indent ‘Guddataa/’Grown up’\\

(10)\\
\indent Dheer-at-aa\\
\indent tall-MID-N\\
\indent ‘Deerataa/Tall guy’\\

\section{Interaction of Agent, Event and Result Nominals}

n some derivation of agent, event and result nominals there seems to be form and semantic relationship between the three. \\

(11)\\
\indent Gudd-is-aa  AGENT\\
\indent Gudd-is-a  Event/Process\\
\indent Gudd-at-aa Result\\


(12)\\
\indent Gabb-is-aa AGENT\\
\indent Gabb-is-a  EVENT/Process\\
\indent Gabb-at-aa RESULT\\

(13)\\
\indent Diim-eess-aa AGENT\\
\indent Diim-ess-a EVENT\\
\indent Diim-at-aa RESULT\\

(14)\\
\indent Bal’-is-aa  AGENT\\
\indent Bal’-is-a  EVENT\\
\indent Bal’-at-aa RESULT\\

(15)\\
\indent Dheer-ess-aa AGENT\\
\indent Dheer-ees-a EVENT\\
\indent Dheer-at-aa RESULT\\




















\newpage

\section{Conclusion}
\end{document}